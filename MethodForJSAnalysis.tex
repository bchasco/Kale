% Options for packages loaded elsewhere
\PassOptionsToPackage{unicode}{hyperref}
\PassOptionsToPackage{hyphens}{url}
%
\documentclass[
]{article}
\usepackage{amsmath,amssymb}
\usepackage{lmodern}
\usepackage{iftex}
\ifPDFTeX
  \usepackage[T1]{fontenc}
  \usepackage[utf8]{inputenc}
  \usepackage{textcomp} % provide euro and other symbols
\else % if luatex or xetex
  \usepackage{unicode-math}
  \defaultfontfeatures{Scale=MatchLowercase}
  \defaultfontfeatures[\rmfamily]{Ligatures=TeX,Scale=1}
\fi
% Use upquote if available, for straight quotes in verbatim environments
\IfFileExists{upquote.sty}{\usepackage{upquote}}{}
\IfFileExists{microtype.sty}{% use microtype if available
  \usepackage[]{microtype}
  \UseMicrotypeSet[protrusion]{basicmath} % disable protrusion for tt fonts
}{}
\makeatletter
\@ifundefined{KOMAClassName}{% if non-KOMA class
  \IfFileExists{parskip.sty}{%
    \usepackage{parskip}
  }{% else
    \setlength{\parindent}{0pt}
    \setlength{\parskip}{6pt plus 2pt minus 1pt}}
}{% if KOMA class
  \KOMAoptions{parskip=half}}
\makeatother
\usepackage{xcolor}
\usepackage[margin=1in]{geometry}
\usepackage{color}
\usepackage{fancyvrb}
\newcommand{\VerbBar}{|}
\newcommand{\VERB}{\Verb[commandchars=\\\{\}]}
\DefineVerbatimEnvironment{Highlighting}{Verbatim}{commandchars=\\\{\}}
% Add ',fontsize=\small' for more characters per line
\usepackage{framed}
\definecolor{shadecolor}{RGB}{248,248,248}
\newenvironment{Shaded}{\begin{snugshade}}{\end{snugshade}}
\newcommand{\AlertTok}[1]{\textcolor[rgb]{0.94,0.16,0.16}{#1}}
\newcommand{\AnnotationTok}[1]{\textcolor[rgb]{0.56,0.35,0.01}{\textbf{\textit{#1}}}}
\newcommand{\AttributeTok}[1]{\textcolor[rgb]{0.77,0.63,0.00}{#1}}
\newcommand{\BaseNTok}[1]{\textcolor[rgb]{0.00,0.00,0.81}{#1}}
\newcommand{\BuiltInTok}[1]{#1}
\newcommand{\CharTok}[1]{\textcolor[rgb]{0.31,0.60,0.02}{#1}}
\newcommand{\CommentTok}[1]{\textcolor[rgb]{0.56,0.35,0.01}{\textit{#1}}}
\newcommand{\CommentVarTok}[1]{\textcolor[rgb]{0.56,0.35,0.01}{\textbf{\textit{#1}}}}
\newcommand{\ConstantTok}[1]{\textcolor[rgb]{0.00,0.00,0.00}{#1}}
\newcommand{\ControlFlowTok}[1]{\textcolor[rgb]{0.13,0.29,0.53}{\textbf{#1}}}
\newcommand{\DataTypeTok}[1]{\textcolor[rgb]{0.13,0.29,0.53}{#1}}
\newcommand{\DecValTok}[1]{\textcolor[rgb]{0.00,0.00,0.81}{#1}}
\newcommand{\DocumentationTok}[1]{\textcolor[rgb]{0.56,0.35,0.01}{\textbf{\textit{#1}}}}
\newcommand{\ErrorTok}[1]{\textcolor[rgb]{0.64,0.00,0.00}{\textbf{#1}}}
\newcommand{\ExtensionTok}[1]{#1}
\newcommand{\FloatTok}[1]{\textcolor[rgb]{0.00,0.00,0.81}{#1}}
\newcommand{\FunctionTok}[1]{\textcolor[rgb]{0.00,0.00,0.00}{#1}}
\newcommand{\ImportTok}[1]{#1}
\newcommand{\InformationTok}[1]{\textcolor[rgb]{0.56,0.35,0.01}{\textbf{\textit{#1}}}}
\newcommand{\KeywordTok}[1]{\textcolor[rgb]{0.13,0.29,0.53}{\textbf{#1}}}
\newcommand{\NormalTok}[1]{#1}
\newcommand{\OperatorTok}[1]{\textcolor[rgb]{0.81,0.36,0.00}{\textbf{#1}}}
\newcommand{\OtherTok}[1]{\textcolor[rgb]{0.56,0.35,0.01}{#1}}
\newcommand{\PreprocessorTok}[1]{\textcolor[rgb]{0.56,0.35,0.01}{\textit{#1}}}
\newcommand{\RegionMarkerTok}[1]{#1}
\newcommand{\SpecialCharTok}[1]{\textcolor[rgb]{0.00,0.00,0.00}{#1}}
\newcommand{\SpecialStringTok}[1]{\textcolor[rgb]{0.31,0.60,0.02}{#1}}
\newcommand{\StringTok}[1]{\textcolor[rgb]{0.31,0.60,0.02}{#1}}
\newcommand{\VariableTok}[1]{\textcolor[rgb]{0.00,0.00,0.00}{#1}}
\newcommand{\VerbatimStringTok}[1]{\textcolor[rgb]{0.31,0.60,0.02}{#1}}
\newcommand{\WarningTok}[1]{\textcolor[rgb]{0.56,0.35,0.01}{\textbf{\textit{#1}}}}
\usepackage{graphicx}
\makeatletter
\def\maxwidth{\ifdim\Gin@nat@width>\linewidth\linewidth\else\Gin@nat@width\fi}
\def\maxheight{\ifdim\Gin@nat@height>\textheight\textheight\else\Gin@nat@height\fi}
\makeatother
% Scale images if necessary, so that they will not overflow the page
% margins by default, and it is still possible to overwrite the defaults
% using explicit options in \includegraphics[width, height, ...]{}
\setkeys{Gin}{width=\maxwidth,height=\maxheight,keepaspectratio}
% Set default figure placement to htbp
\makeatletter
\def\fps@figure{htbp}
\makeatother
\setlength{\emergencystretch}{3em} % prevent overfull lines
\providecommand{\tightlist}{%
  \setlength{\itemsep}{0pt}\setlength{\parskip}{0pt}}
\setcounter{secnumdepth}{-\maxdimen} % remove section numbering
\usepackage{booktabs}
\usepackage{longtable}
\usepackage{array}
\usepackage{multirow}
\usepackage{wrapfig}
\usepackage{float}
\usepackage{colortbl}
\usepackage{pdflscape}
\usepackage{tabu}
\usepackage{threeparttable}
\usepackage{threeparttablex}
\usepackage[normalem]{ulem}
\usepackage{makecell}
\usepackage{xcolor}
\ifLuaTeX
  \usepackage{selnolig}  % disable illegal ligatures
\fi
\IfFileExists{bookmark.sty}{\usepackage{bookmark}}{\usepackage{hyperref}}
\IfFileExists{xurl.sty}{\usepackage{xurl}}{} % add URL line breaks if available
\urlstyle{same} % disable monospaced font for URLs
\hypersetup{
  pdftitle={Methods for JS Lewis analysis},
  pdfauthor={Brandon},
  hidelinks,
  pdfcreator={LaTeX via pandoc}}

\title{Methods for JS Lewis analysis}
\author{Brandon}
\date{2025-02-20}

\begin{document}
\maketitle

\hypertarget{methods}{%
\section{Methods}\label{methods}}

This is an individual based model describing the estimate of carcass
abundance in the Lewis River. This model is different than that of
Schwarz. The goal is to estimate the total carcass abundance (i.e.,
\(\lambda\) \ref{GenericLikelihood})

To understand the likelihood, it is best to start with a model that has
no spatial component. \begin{equation} \label{eq:Likelihood}
  L = P(B_{total}|\lambda p)\prod_i^{B_{total}} P(t_i|\boldsymbol{\pi})P(D_{i,t}|t_i,\phi_i,p_i)P(T_{i,t}|D_{i,t},p_i)P(R_{i,t\prime}|T_{i,t},\phi_i,p_i) 
\end{equation}

Breaking down the likelihood into it's components, the probability of
the observed total number of carcasses is
\begin{equation}\label{P_PopulationSize}
  P(B_{total}|\lambda p)
\end{equation} where, \(\lambda\) is total population size, and p is the
capture probability of a carcass. The probability of the arrival process
is, \begin{equation}\label{P_arrival}
  P(t_i|\boldsymbol{\pi})
\end{equation} where, \(t_i\) is a latent effect, and
\(\boldsymbol{\pi}\) vector of arrival probabilities for each time-step.
The probability of the initial detection for any carcass,
\begin{equation}\label{P_initial detection}
  P(D_{i,t}|t_i,\phi_i,p_i)
\end{equation} where, \(D_{i,t} = 1\), \(t_i\) is the latent estimate of
arrival time, \(phi_i\) is the survival probability, and \(p_i\) is the
detection probability. The probability of carcass being tagged is,
\begin{equation}\label{P_tagging}
  P(T_{i,t}|D_{i,t},\psi)
\end{equation} where, \(T_{i,t}=1\) is a carcass was tagged, \(\psi\) is
the tagging rate. Finally, there is probability of recapture,

\begin{equation}\label{P_recapture}
  P(R_{i,t\prime}|T_{i,t},\phi_i,p_i)
\end{equation}

where, \(R_{i,t\prime} = 1\) for a carcass recaptured on day
\(t\prime\).

\hypertarget{create-the-data}{%
\subsection{Create the data}\label{create-the-data}}

We begin by reading the data and transforming some of the output to make
it easier to work in RTMB.

\begin{Shaded}
\begin{Highlighting}[]
\CommentTok{\#Data list}
\FunctionTok{library}\NormalTok{(dplyr)}
\end{Highlighting}
\end{Shaded}

\begin{verbatim}
## 
## Attaching package: 'dplyr'
\end{verbatim}

\begin{verbatim}
## The following objects are masked from 'package:stats':
## 
##     filter, lag
\end{verbatim}

\begin{verbatim}
## The following objects are masked from 'package:base':
## 
##     intersect, setdiff, setequal, union
\end{verbatim}

\begin{Shaded}
\begin{Highlighting}[]
\FunctionTok{library}\NormalTok{(tidyr)}

\NormalTok{d }\OtherTok{\textless{}{-}} \FunctionTok{read.csv}\NormalTok{(}\StringTok{"data/simpleData2.csv"}\NormalTok{) }\SpecialCharTok{\%\textgreater{}\%}
  \FunctionTok{mutate}\NormalTok{(}\AttributeTok{t\_wk =}\NormalTok{ lubridate}\SpecialCharTok{::}\FunctionTok{week}\NormalTok{(lubridate}\SpecialCharTok{::}\FunctionTok{mdy}\NormalTok{(TagDate)),}
         \AttributeTok{r\_wk =}\NormalTok{ lubridate}\SpecialCharTok{::}\FunctionTok{week}\NormalTok{(lubridate}\SpecialCharTok{::}\FunctionTok{mdy}\NormalTok{(RecapDate)),}
         \AttributeTok{t\_yr =}\NormalTok{ lubridate}\SpecialCharTok{::}\FunctionTok{year}\NormalTok{(lubridate}\SpecialCharTok{::}\FunctionTok{mdy}\NormalTok{(TagDate)),}
         \AttributeTok{r\_yr =}\NormalTok{ lubridate}\SpecialCharTok{::}\FunctionTok{year}\NormalTok{(lubridate}\SpecialCharTok{::}\FunctionTok{mdy}\NormalTok{(RecapDate))) }\SpecialCharTok{\%\textgreater{}\%}
  \FunctionTok{filter}\NormalTok{(t\_yr }\SpecialCharTok{==} \DecValTok{2024}\NormalTok{) }\SpecialCharTok{\%\textgreater{}\%} 
  \FunctionTok{filter}\NormalTok{(t\_wk }\SpecialCharTok{\textgreater{}} \DecValTok{10}\NormalTok{) }\SpecialCharTok{\%\textgreater{}\%}
  \FunctionTok{mutate}\NormalTok{(}\AttributeTok{t\_k =}\NormalTok{ t\_wk }\SpecialCharTok{{-}} \FunctionTok{min}\NormalTok{(t\_wk) }\SpecialCharTok{+} \DecValTok{1}\NormalTok{,}
         \AttributeTok{r\_k =}\NormalTok{ r\_wk }\SpecialCharTok{{-}} \FunctionTok{min}\NormalTok{(t\_wk) }\SpecialCharTok{+} \DecValTok{1}\NormalTok{) }\SpecialCharTok{\%\textgreater{}\%}
  \FunctionTok{mutate}\NormalTok{(}\AttributeTok{t\_l =}\NormalTok{ TagState,}
         \AttributeTok{r\_l =}\NormalTok{ RecapState) }\SpecialCharTok{\%\textgreater{}\%}
  \FunctionTok{filter}\NormalTok{(}\FunctionTok{is.na}\NormalTok{(r\_wk) }\SpecialCharTok{|}\NormalTok{ r\_k}\SpecialCharTok{\textgreater{}}\DecValTok{0}\NormalTok{) }\SpecialCharTok{\%\textgreater{}\%} 
  \FunctionTok{mutate}\NormalTok{(}\AttributeTok{tag =} \FunctionTok{ifelse}\NormalTok{(Tag1}\SpecialCharTok{==}\StringTok{""}\NormalTok{,}\ConstantTok{FALSE}\NormalTok{,}\ConstantTok{TRUE}\NormalTok{)) }\SpecialCharTok{\%\textgreater{}\%}
  \FunctionTok{group\_by}\NormalTok{(t\_k,r\_k,t\_l,r\_l,tag) }\SpecialCharTok{\%\textgreater{}\%}
  \FunctionTok{summarise}\NormalTok{(}\AttributeTok{n =} \FunctionTok{n}\NormalTok{())}
\end{Highlighting}
\end{Shaded}

\begin{verbatim}
## `summarise()` has grouped output by 't_k', 'r_k', 't_l', 'r_l'. You can
## override using the `.groups` argument.
\end{verbatim}

\hypertarget{just-model-the-cjs-part-of-the-likelihood}{%
\subsection{Just model the CJS part of the
likelihood}\label{just-model-the-cjs-part-of-the-likelihood}}

It is possible to breaks down the analysis in to separate components. We
can start by just looking at the Cormack-Jolly-Seber part of the model.
For those tagged individuals that are tagged, we can estimate survival
and detection probability. For now, we can simply assume all fish are
equal and the detection and survival is constant across time and
location.

I am going to use a matrix-algebra approach to analyzing the data. And
start with the simplest data set possible.

\begin{Shaded}
\begin{Highlighting}[]
\NormalTok{data }\OtherTok{\textless{}{-}} \FunctionTok{list}\NormalTok{(}\AttributeTok{t\_l =}\NormalTok{ d}\SpecialCharTok{$}\NormalTok{t\_l, }\CommentTok{\#tagging location}
             \AttributeTok{r\_l =}\NormalTok{ d}\SpecialCharTok{$}\NormalTok{r\_l, }\CommentTok{\#recapture location}
             \AttributeTok{t\_k =}\NormalTok{ d}\SpecialCharTok{$}\NormalTok{t\_k, }\CommentTok{\#tagging week}
             \AttributeTok{r\_k =}\NormalTok{ d}\SpecialCharTok{$}\NormalTok{r\_k,}
             \AttributeTok{tag =}\NormalTok{ d}\SpecialCharTok{$}\NormalTok{tag,}
             \AttributeTok{n =}\NormalTok{ d}\SpecialCharTok{$}\NormalTok{n) }\CommentTok{\#recapture week, last week if not recapture}

\CommentTok{\# Initial parameter values}
\NormalTok{parameters }\OtherTok{\textless{}{-}} \FunctionTok{list}\NormalTok{(}
  \AttributeTok{phi\_par =} \DecValTok{0}\NormalTok{,}
  \AttributeTok{p\_par =} \DecValTok{0}
\NormalTok{  )}
\end{Highlighting}
\end{Shaded}

Now lets create the likelihoods function given the data.

\begin{Shaded}
\begin{Highlighting}[]
\NormalTok{f }\OtherTok{\textless{}{-}} \ControlFlowTok{function}\NormalTok{(parms)\{}

\NormalTok{  RTMB}\SpecialCharTok{::}\FunctionTok{getAll}\NormalTok{(data,}
\NormalTok{               parms)}
  
  \CommentTok{\#Negative}
\NormalTok{  nll }\OtherTok{\textless{}{-}} \FunctionTok{rep}\NormalTok{(}\DecValTok{0}\NormalTok{,}\FunctionTok{length}\NormalTok{(d}\SpecialCharTok{$}\NormalTok{t\_l))}
  
  \CommentTok{\#Survival}
\NormalTok{  phi }\OtherTok{\textless{}{-}} \FunctionTok{matrix}\NormalTok{(}\DecValTok{0}\NormalTok{,}\DecValTok{2}\NormalTok{,}\DecValTok{2}\NormalTok{)}
\NormalTok{  phi[}\DecValTok{1}\NormalTok{,}\DecValTok{1}\NormalTok{] }\OtherTok{\textless{}{-}} \SpecialCharTok{{-}}\FunctionTok{exp}\NormalTok{(phi\_par)}
\NormalTok{  phi[}\DecValTok{1}\NormalTok{,}\DecValTok{2}\NormalTok{] }\OtherTok{\textless{}{-}} \SpecialCharTok{{-}}\NormalTok{phi[}\DecValTok{1}\NormalTok{,}\DecValTok{1}\NormalTok{]}
\NormalTok{  phi }\OtherTok{\textless{}{-}}\NormalTok{ Matrix}\SpecialCharTok{::}\FunctionTok{expm}\NormalTok{(phi)}
  
  \CommentTok{\#Detection probability}
\NormalTok{  p }\OtherTok{\textless{}{-}} \FunctionTok{matrix}\NormalTok{(}\DecValTok{0}\NormalTok{,}\DecValTok{2}\NormalTok{,}\DecValTok{2}\NormalTok{)}
\NormalTok{  p[}\DecValTok{1}\NormalTok{,}\DecValTok{1}\NormalTok{] }\OtherTok{\textless{}{-}} \SpecialCharTok{{-}}\FunctionTok{exp}\NormalTok{(p\_par)}
\NormalTok{  p[}\DecValTok{1}\NormalTok{,}\DecValTok{2}\NormalTok{] }\OtherTok{\textless{}{-}} \SpecialCharTok{{-}}\NormalTok{p[}\DecValTok{1}\NormalTok{,}\DecValTok{1}\NormalTok{]}
\NormalTok{  p }\OtherTok{\textless{}{-}}\NormalTok{ Matrix}\SpecialCharTok{::}\FunctionTok{expm}\NormalTok{(p)}
  
  \CommentTok{\#This is the amount of time to first detection}
  \ControlFlowTok{for}\NormalTok{(i }\ControlFlowTok{in} \DecValTok{1}\SpecialCharTok{:}\FunctionTok{length}\NormalTok{(t\_l))\{}
    
    \CommentTok{\#Accumulator}
\NormalTok{    m }\OtherTok{\textless{}{-}} \FunctionTok{matrix}\NormalTok{(}\DecValTok{0}\NormalTok{,}\DecValTok{2}\NormalTok{,}\DecValTok{2}\NormalTok{)}
    \FunctionTok{diag}\NormalTok{(m) }\OtherTok{\textless{}{-}} \DecValTok{1}
    
    \CommentTok{\#Initial state}
\NormalTok{    delta }\OtherTok{\textless{}{-}} \FunctionTok{rep}\NormalTok{(}\DecValTok{0}\NormalTok{,}\DecValTok{2}\NormalTok{)}
\NormalTok{    delta[}\DecValTok{1}\NormalTok{] }\OtherTok{\textless{}{-}} \DecValTok{1}
    \ControlFlowTok{if}\NormalTok{(tag[i] }\SpecialCharTok{==} \ConstantTok{TRUE}\NormalTok{ )\{}
      \ControlFlowTok{if}\NormalTok{(}\FunctionTok{is.na}\NormalTok{(r\_k[i]))\{}
\NormalTok{        Ui }\OtherTok{\textless{}{-}} \DecValTok{16}
\NormalTok{      \}}\ControlFlowTok{else}\NormalTok{\{}
\NormalTok{        Ui }\OtherTok{\textless{}{-}}\NormalTok{ r\_k[i]}
\NormalTok{      \}}
      \ControlFlowTok{for}\NormalTok{(j }\ControlFlowTok{in}\NormalTok{ (t\_k[i]}\SpecialCharTok{+}\DecValTok{1}\NormalTok{)}\SpecialCharTok{:}\NormalTok{Ui)\{}
        \ControlFlowTok{if}\NormalTok{(j}\SpecialCharTok{\textless{}}\NormalTok{Ui)\{ }\CommentTok{\#not detected}
\NormalTok{          m }\OtherTok{\textless{}{-}}\NormalTok{ m }\SpecialCharTok{\%*\%}\NormalTok{ phi }\SpecialCharTok{\%*\%} \FunctionTok{diag}\NormalTok{(p[,}\DecValTok{2}\NormalTok{])}
\NormalTok{        \}}\ControlFlowTok{else}\NormalTok{\{}\CommentTok{\#last observation}
          \ControlFlowTok{if}\NormalTok{(}\FunctionTok{is.na}\NormalTok{(r\_k[i]))\{}\CommentTok{\#not detected}
\NormalTok{            m }\OtherTok{\textless{}{-}}\NormalTok{ m }\SpecialCharTok{\%*\%}\NormalTok{ phi }\SpecialCharTok{\%*\%} \FunctionTok{diag}\NormalTok{(p[,}\DecValTok{2}\NormalTok{])}
\NormalTok{          \}}\ControlFlowTok{else}\NormalTok{\{}\CommentTok{\#Detected}
\NormalTok{            m }\OtherTok{\textless{}{-}}\NormalTok{ m }\SpecialCharTok{\%*\%}\NormalTok{ phi }\SpecialCharTok{\%*\%} \FunctionTok{diag}\NormalTok{(p[,}\DecValTok{1}\NormalTok{])}
\NormalTok{          \}}
\NormalTok{        \}}
\NormalTok{      \}}
\NormalTok{    \}}
\NormalTok{    nll[i] }\OtherTok{\textless{}{-}} \FunctionTok{log}\NormalTok{(}\FunctionTok{t}\NormalTok{(delta) }\SpecialCharTok{\%*\%}\NormalTok{ m }\SpecialCharTok{\%*\%} \FunctionTok{rep}\NormalTok{(}\DecValTok{1}\NormalTok{,}\DecValTok{2}\NormalTok{))}
\NormalTok{  \}}

\NormalTok{  RTMB}\SpecialCharTok{::}\FunctionTok{REPORT}\NormalTok{(nll)  }
\NormalTok{  RTMB}\SpecialCharTok{::}\FunctionTok{REPORT}\NormalTok{(phi)  }
\NormalTok{  RTMB}\SpecialCharTok{::}\FunctionTok{REPORT}\NormalTok{(p)  }
  \FunctionTok{return}\NormalTok{(}\SpecialCharTok{{-}}\FunctionTok{sum}\NormalTok{(nll}\SpecialCharTok{*}\NormalTok{n))}
  \CommentTok{\# return(0)}
\NormalTok{\}}
\end{Highlighting}
\end{Shaded}

Next we can optimize.

\begin{verbatim}
## outer mgc:  964.4787 
## outer mgc:  477.1059 
## outer mgc:  397.2693 
## outer mgc:  83.36825 
## outer mgc:  115.5598 
## outer mgc:  23.24397 
## outer mgc:  4.240043 
## outer mgc:  6.129297 
## outer mgc:  6.249639 
## outer mgc:  1.467554 
## outer mgc:  2.381205 
## outer mgc:  0.07488745 
## outer mgc:  0.0007851369 
## outer mgc:  1.521512e-05
\end{verbatim}

Finally, we can look at the results of \(\phi\) and \(p\) matrixes,

\begin{Shaded}
\begin{Highlighting}[]
\CommentTok{\#Survival/persistence}
\FunctionTok{print}\NormalTok{(}\FunctionTok{round}\NormalTok{(rep}\SpecialCharTok{$}\NormalTok{phi,}\DecValTok{2}\NormalTok{))}
\end{Highlighting}
\end{Shaded}

\begin{verbatim}
## 2 x 2 Matrix of class "dgeMatrix"
##      [,1] [,2]
## [1,] 0.51 0.49
## [2,] 0.00 1.00
\end{verbatim}

\begin{Shaded}
\begin{Highlighting}[]
\CommentTok{\#Survival/persistence}
\FunctionTok{print}\NormalTok{(}\FunctionTok{round}\NormalTok{(rep}\SpecialCharTok{$}\NormalTok{p,}\DecValTok{2}\NormalTok{))}
\end{Highlighting}
\end{Shaded}

\begin{verbatim}
## 2 x 2 Matrix of class "dgeMatrix"
##      [,1] [,2]
## [1,] 0.61 0.39
## [2,] 0.00 1.00
\end{verbatim}

In this simple model, for the tagged individuals, the
``persistence/survival'' between time steps is 0.51 and the detection
probability is 0.61.

\hypertarget{just-model-the-cjs-part-of-the-likelihood-1}{%
\subsection{Just model the CJS part of the
likelihood}\label{just-model-the-cjs-part-of-the-likelihood-1}}

\hypertarget{tables}{%
\section{Tables}\label{tables}}

\end{document}
